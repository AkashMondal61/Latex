
    \begin{conf-abstract}[]
        {\textbf{Unsupervised MTS Anomaly Detection with Variational AutoEncoders}}
        {\textit{Saravana  M K$^{1}$, Roopa M S$^{2}$}}
        {$^{1}$University Visvesvaraya College of Engineering}
        {\texttt{mksaravanamk1@gmail.com, roopams22@gmail.com}}
        \indexauthors{K!Saravana  M, S!Roopa M}
        {MTS data often involves multiple variables or measurements recorded at each time point which poses several challenges, such as High-Dimensionality, which increases the complexity of analyzing and detecting anomalies, as the interactions between variables need to be considered. Handling high-dimensional data requires careful feature selection, dimensionality reduction techniques, or models specifically designed to handle such data. In addition, MTS data exhibits temporal dependencies, i.e, observations at different time points are correlated. However, the strength and nature of these dependencies can vary, making it challenging accurately to model and capture the underlying patterns. Detecting anomalies requires understanding the temporal dynamics and distinguishing between normal variations and unusual behaviors. Most importantly, in many real-world scenarios, labeled anomalies are scarce or non-existent. Anomaly detection methods often rely on labeled data for training and evaluation, but without sufficient labeled anomalies, it becomes difficult to develop effective models. Unsupervised or semi-supervised techniques are often employed in such cases, but their performance is limited compared to supervised methods. Addressing these challenges requires domain knowledge, appropriate feature engineering, careful algorithm selection, and model evaluation. Additionally, iterative refinement and adaptation of anomaly detection methods based on feedback and domain-specific insights are often necessary for achieving optimal results. In this paper we present Unsupervised Anomaly Detection model for MTS data based on Variational Autoencoder called as VAE-MTS which outperforms traditional Autoencoders and GAN-based Anomaly detection methods.}
    \end{conf-abstract}
        