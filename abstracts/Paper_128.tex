
    \begin{conf-abstract}[]
        {\textbf{Trust Management Model for Service Delegation in SIoT}}
        {\textit{Santhosh Kumari$^{1}$}}
        {$^{1}$UVCE, Bangalore}
        {\texttt{santhoshkumariy@gmail.com}}
        \indexauthors{Kumari!Santhosh}
        {The Social Internet of Things (SIoT) is an extension of Internet of Things that blends social networking ideas and concepts into the domain of social networking, device connectivity and communication. During exchange of information and services with their peers, trust plays an important role in establishing relationship with trustworthy peers and reduces the risk of being exposed to malicious nodes. The existing approaches lack adequate trust quantification and are further hindered by biased evaluations of trustworthiness, as they fail to consider the diverse behavior of a node when delivering different services. In this paper, a trust Management model is introduced to calculate trust between communicating nodes in relation to a particular service. Additionally, an algorithm is proposed to recommend a more reliable service provider for service delegation, aiming to reduce the vulnerability to malicious nodes. The trust management model incorporates various factors, including direct trust, recommended trust, co-operativeness, computation capability, and relationship factors, to capture the behavior of a node. These attributes are aggregated using the MLP-R algorithm to generate a consolidated trust score for each node. The proposed trust model is dependable and effective in boosting the success rate of services, while simultaneously enhancing the efficiency and security of services within the realm of SIoT.}
    \end{conf-abstract}
        