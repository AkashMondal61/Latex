
    \begin{conf-abstract}[]
        {\textbf{On Coverage and Connectivity with Reduced Sensing Redundancy based on Genetic Algorithm}}
        {\textit{Rajib  Kumar Mondal$^{1}$, Sanghita Bhattacharjee$^{2}$, Tandra Pal$^{3}$}}
        {$^{1}$Dr. B.C. Roy Engineering College $\bullet$ $^{2}$National Institute of Technology, Durgapur $\bullet$ $^{3}$NIT Durgapur}
        {\texttt{rajib94@gmail.com, sanghita.bhattacharjee@cse.nitdgp.ac.in, tandra.pal@cse.nitdgp.ac.in}}
        \indexauthors{Mondal!Rajib  Kumar, Bhattacharjee!Sanghita, Pal!Tandra}
        {In Wireless Sensor Networks (WSNs), coverage and connectivity are crucial factors for effectively monitoring targets and transmitting data to the base station. To increase the network's robustness, targets must be covered by multiple sensors. In k-coverage, each target is monitored by a minimum of k distinct sensors. Many studies focus on achieving k -coverage but they overlook the issue of sensing redundancy. Sensing redundancy occurs when targets are covered by more than k sensors. However, minimizing redundant sensing may lead to coverage holes in the network. To address this problem, our study proposes a genetic algorithm that considers four conflicting objectives: minimizing the number of sensors, maximizing k -coverage, maximizing m -connectivity, and minimizing sensing redundancy. The fitness function is designed to balance these objectives. By reducing the number of targets that generates sensing redundancy, our proposed method aims to minimize redundant sensing and network energy consumption. Simulation results demonstrate the efficacy of our algorithm, which outperforms two existing works in the literature}
    \end{conf-abstract}
        