
    \begin{conf-abstract}[]
        {\textbf{A Study on Users Sentiment from Twitter Data and Stock Market During Russia-Ukraine War }}
        {\textit{Sutapa Bhattacharya$^{1}$, Gunjan Kumar Biswas$^{2}$, Bibek  Roy$^{3}$, Dipak kole$^{4}$, Dhrubasish Sarkar$^{5}$, Koushik Majumder$^{6}$}}
        {$^{1}$SILIGURI INSTITUTE OF TECHNOLOGY $\bullet$ $^{2}$Jalpaiguri Government Engineering College $\bullet$ $^{3}$Jalpaiguri Government Engineering College $\bullet$ $^{4}$jgec $\bullet$ $^{5}$Supreme Institute of Management and Technology $\bullet$ $^{6}$Maulana Abul Kalam Azad University of Technology, WB}
        {\texttt{sutapa2007@gmail.com, gk2252@cse.jgec.ac.in, br2256@cse.jgec.ac.in, dipak.kole@cse.jgec.ac.in, dhrubasish@inbox.com, koushikzone@yahoo.com}}
        \indexauthors{Bhattacharya!Sutapa, Biswas!Gunjan Kumar, Roy!Bibek , kole!Dipak, Sarkar!Dhrubasish, Majumder!Koushik}
        {The stock market depends on several factors such as currency exchange rates, current affairs, supply and demand. Stock market values can fluctuate a lot on the people's opinions, and their opinions on social media can provide useful information for analysis. In this paper, tweets are first extracted with the subject of war and the sentiment of those are calculated, then the tweets posted during the war period between Russia and Ukraine related to stocks are extracted and sentiment analysis is performed on the Twitter data. The result obtained from Twitter is compared with the real-time stock exchange sites such as Nifty and Sensex. So, the main aim of this paper is to verify whether the war has imposed a negative impact on people's mindsets which in turn has negatively affected the stock market.}
    \end{conf-abstract}
        