
    \begin{conf-abstract}[]
        {\textbf{Securing Social Internet of Things: Intrusion Detection Models in Collaborative Edge Computing }}
        {\textit{Divya  S$^{1}$, Manjula S H$^{2}$, Dr. Venugopal K. R.$^{3}$}}
        {$^{1}$UVCE Bangalore university $\bullet$ $^{2}$University Visveswarya College of Engg. K R Circle, Bangalore-560001 $\bullet$ $^{3}$B.N.M. Institute of Technology}
        {\texttt{divyasdivu1994@gmail.com, shmanjula@gmail.com, venugopalkr@gmail.com}}
        \indexauthors{S!Divya , H!Manjula S, R.!Dr. Venugopal K.}
        {The rapid integration of the Social Internet of Things (SIoT) into our everyday lives necessitates robust security measures. Collaborative Edge Computing (CEC) has emerged as a promising solution to mitigate resource congestion in SIoT environments. However, ensuring the security of edge networks remains a critical challenge due to various types of attacks and unauthorized access. In this paper, we propose a comparative analysis of intrusion detection models, including Generative Adversarial Networks (GAN), Convolutional Neural Networks (CNN), and Logistic Regression (LR), for CEC-based SIoT. Our study encompasses three major components: feature extraction, model design, and evaluation. We evaluate the performance of each model using the CSE-CIC-IDS 2018 and CIC-DDoS 2019 datasets, considering metrics such as accuracy, precision, recall, and F1 score. The results demonstrate that our CNN-based intrusion detection model achieves better accuracy compared to GAN and LR. Additionally, we outline future research directions, including the integration of CNN and GAN for spatiotemporal feature extraction and the development of real-time feature extraction algorithms. By shedding light on the effectiveness of different intrusion detection models, our work contributes to enhancing the security of SIoT in CEC environments, leading to safer and more reliable IoT deployments.}
    \end{conf-abstract}
        