
    \begin{conf-abstract}[]
        {\textbf{Vessel Curvature based Data Augmentation Technique for Retinal Fundus Images}}
        {\textit{Supratim Ghosh$^{1}$, Sourav Pramanik$^{2}$, Prof. Mita Nasipuri$^{3}$, Prof. Mahantapas Kundu$^{4}$}}
        {$^{1}$Institute of Engineering \& Management $\bullet$ $^{2}$New Alipore College $\bullet$ $^{3}$Jadavpur University $\bullet$ $^{4}$Jadavpur University}
        {\texttt{supratimghosh2772@gmail.com, srv.pramanik03327@gmail.com, mitanasipuri@gmail.com, mahantapas@gmail.com}}
        \indexauthors{Ghosh!Supratim, Pramanik!Sourav, Nasipuri!Prof. Mita, Kundu!Prof. Mahantapas}
        {A novel vessel structure based data set augmentation approach has been proposed in this work to aid the learning of a vanilla U-Net based model for Retinal Vessel Segmentation. A functional formulation is proposed in this work to represent a Fundus Image as a mix of background surface and vessel surface. We have individually proposed a separate model for operating the background region and vessel curves independently and validated the augmentation strength of the proposed work by testing the performance of the learned U-Net model on two benchmark datasets, namely, DRIVE and STARE. Finally, we have compared and reported the performance of the U-Net model, trained on the augmented data, based on the metrics of Accuracy, Specificity and Sensitivity in this work.}
    \end{conf-abstract}
        