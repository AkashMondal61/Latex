
    \begin{conf-abstract}[]
        {\textbf{MARS: Manual \& Automatic Robotic Sanitization on Social Milieu }}
        {\textit{Raushan  Kumar Singh$^{1}$}}
        {$^{1}$IIT ROPAR}
        {\texttt{Raushan.21csz0020@iitrpr.ac.in}}
        \indexauthors{Singh!Raushan  Kumar}
        {Sanitization is not a new term, but with the evolution of deadly COVID-19, the process came in limelight in no time. The process was already utilized widely in hospitals, vaccination centres, food processing units and medicine industries, suddenly became crucial in every domain related to our lives. Even though sanitization is considered the first line of defence against pan-demic viruses like COVID-19, it is highly difficult to sanitize every nook and corner of bigger buildings and external structures like Airports, Railway sta-tions, theatres, institutions and hospitals. Slight carelessness to eliminate the virus from the sanitization process can reciprocate in the pandemic spread. Our proposed work deals with utilizing the accuracy and precision of robots to effectively sanitize the bigger structures. The multi-faceted methodology of the work manages the comprehensive investigation of the robotic unit for the social setting. The concentrate additionally stretches out to refine the standard human behavioural reaction for modern robotic consideration in our lives. This will ease up the process and at the same time, will reduce the chance of human error. The robotic structure is powered by a 12V recharge-able battery, which has manual and automation modes of cleaning. During manual mode, we control the robot with an android application installed on the phone and connected with the robot through Bluetooth wireless connec-tivity. During automation, the mode robot moves in different directions and cleans \& sanitizes the area on its own. There is an ESP8266-based IoT con-nection unit to update the overall process for the cloud. }
    \end{conf-abstract}
        