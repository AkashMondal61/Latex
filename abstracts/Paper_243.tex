
    \begin{conf-abstract}[]
        {\textbf{Emotion Detection Using Pattern Recognition from Speech}}
        {\textit{Arijit Ghosal$^{1}$, Harshita  Somolu$^{2}$, Suchibrota  Dutta$^{3}$}}
        {$^{1}$St. Thomas' College of Engineering \& Technology $\bullet$ $^{2}$Cognizant Technology Solutions $\bullet$ $^{3}$Royal Thimphu College, Thimphu, Bhutan}
        {\texttt{arijitghosal.official@gmail.com, harshita.calling@gmail.com,  suchibrota@gmail.com}}
        \indexauthors{Ghosal!Arijit, Somolu!Harshita , Dutta!Suchibrota }
        {Detection of emotion from speech has immense importance in our present life. Emotion detection plays an important role in music therapy of human being. But detection of emotion is a problematic task. The objective of this work is to detect the underlying emotion in the audio speech along with a high accuracy. Emotion is a very important factor in communication. It has an impact in making decisions. In the virtual world, where people depend on artificial approaches, analyzing a person's emotion plays an important role. Thus, objective is to detect the underlying emotion in a the audio speech and assign it to either joy, anger, boredom, sadness, fear, frustration, annoyance, satisfaction or neutral or any other emotion. This work considers neutral, happy, sad and angry emotions only from speech signal. Emotions within speech mostly varies in the frequency domain. MFCC (Mel-Frequency Cepstral Co-efficients) is a very good aural facet for capturing different emotions within speech. MFCC is supported by some other popular frequency domain aural facets as well as some time domain aural facets. ZCR (Zero Crossing Rate) and STE (Short Time Energy) have been used as time domain aural facets, Spectral RollOff, Skewness, Spectral Flux have been used as frequency domain aural facets. For precise observation co-occurrence based facets have also been considered for Spectral Flux as well as MFCC. For classification task Multi-Layer Perceptron (Neural Network), Random Forest, Naïve Bayes and Logistic Regression have been used. Experimental result exhibits superiority of the work.}
    \end{conf-abstract}
        