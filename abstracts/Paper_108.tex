
    \begin{conf-abstract}[]
        {\textbf{Anomaly Detection based Resource Autoscaling Mechanism for Fog Computing}}
        {\textit{Sheela S$^{1}$, Dilip  S M Kumar$^{2}$}}
        {$^{1}$UVCE $\bullet$ $^{2}$UVCE}
        {\texttt{sheelasresearch@gmail.com, dilipkumarsm.uvce@gmail.com}}
        \indexauthors{S!Sheela, Kumar!Dilip  S M}
        {Fog computing leverages the cloud computing services at the network edge. It aims to address the issues of latency, bandwidth constraints, and network reliability in the context of IoT. However, fog computing paradigm faces issues in adopting to the dynamic workloads, optimizing resource utilization and ensuring efficient performance due to the dynamicity in the requests raised by the IoT devices and the edge applications. This work proposes a anomaly detection based resource autoscaling mechanism based on the deep autoencoders. Deep autoencoders uses neural networks to detect the anomalies in the data, based on which the virtual machines in the fog node are auto scaled. Anomaly scores and response times for the three variants of deep autoencoders are compared. The outcome indicates that the proposed approach improves the scalability, responsiveness and resource utilization.}
    \end{conf-abstract}
        