
    \begin{conf-abstract}[]
        {\textbf{Gate-Based Fractal Analysis for Assessing Complexity and Persistence of Mangrove Communities in Remote Sensing Applications}}
        {\textit{Anindita Mrs Das Bhattacharjee$^{1}$, Somdatta  Chakravortty$^{2}$}}
        {$^{1}$IEM $\bullet$ $^{2}$Maulana Abul Kalam Azad University of Technology, West Bengal}
        {\texttt{Anindita.DasBhattacharjee@iem.edu.in, csomdatta@rediffmail.com}}
        \indexauthors{Bhattacharjee!Anindita Mrs Das, Chakravortty!Somdatta }
        {This study focuses on using fractal dimension analysis to assess mangrove habitat complexity in the four segmented Region Of Interests in Henry Island region spanning from 88°13'55" E to 88°18'50" E longitude and 21°32'30" N to 21°38'50" N latitude of the Sundarbans. The evaluation of fractal dimension values reveals distinct mangrove communities in different regions of interest (ROI). The proposed Gate-based fractal dimension evaluation approach is utilized, incorporating techniques like Exponential Dilation and Shadow removal for better edge identification and image noise reduction. Evaluating Fractal dimensions and Hurst exponent values provides insights into spatial heterogeneity, diversity, and long-term persistence of mangrove habitats. The modified gate-based fractal analysis method is employed, offering improved stability compared to box-counting methods. Landsat 8 satellite data from 2019 to 2023 is used for analysis. ROI2 exhibits the highest average fractal dimension value of 1.949, indicating a diverse and heterogeneous habitat. The Hurst exponent values support the presence of different mangrove communities in ROI1 and ROI2, as well as ROI2 and ROI3, highlighting unique communities within each region.The findings enhance knowledge of spatial patterns, diversity, and stability in mangrove ecosystems.}
    \end{conf-abstract}
        