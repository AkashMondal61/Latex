
    \begin{conf-abstract}[]
        {\textbf{Maatran: Revolutionizing Maternal Care through Remote Monitoring and Risk Prediction}}
        {\textit{Kulsum Kamal$^{1}$, Niladri  Shekhar Das$^{2}$, Subroto Rakshit$^{3}$, Rudraneel Dutta$^{4}$, Sovan Saha$^{5}$}}
        {$^{1}$Institute of Engineering and Management $\bullet$ $^{2}$Institute of Engineering and Management $\bullet$ $^{3}$Institute of Engineering and Management $\bullet$ $^{4}$Institute of Engineering and Management $\bullet$ $^{5}$Techno Main Salt Lake}
        {\texttt{kulsumkamal26@gmail.com, boney0310@gmail.com, rakshitsubroto123@gmail.com, rishi18neel@gmail.com, sovansaha12@gmail.com}}
        \indexauthors{Kamal!Kulsum, Das!Niladri  Shekhar, Rakshit!Subroto, Dutta!Rudraneel, Saha!Sovan}
        {Maternal mortality due to pregnancy complications remains a significant global challenge, necessitating the development of innovative approaches to safeguard pregnant women from potential threats. This paper introduces Maatran, a technological solution that leverages IoT, Cloud technologies, and Machine Learning solutions to address maternal health issues. Maatran utilizes Arduino-based wearable sensor devices to remotely collect health data, which is then transmitted to an Android app interface for data processing and storage in the cloud. A Random Forest model is employed to accurately predict maternal health risks. The system's features provide a holistic solution for remote health monitoring of pregnant women and facilitate direct communication with healthcare professionals. Through rigorous machine learning model training, Maatran achieved accuracy, roc-auc and log loss of 0.93, 0.91 and 0.93 respectively, surpassing existing state-of-the-art models. The implementation of Maatran demonstrates its potential to significantly improve maternal health outcomes by enabling early risk detection and timely interventions.}
    \end{conf-abstract}
        