
    \begin{conf-abstract}[]
        {\textbf{Modeling the Role of Gap Junctions in An Olfactory Neuropil, The Antennal Lobe }}
        {\textit{Dileep G$^{1}$}}
        {$^{1}$IIT Mandi}
        {\texttt{dileepmon2@gmail.com}}
        \indexauthors{G!Dileep}
        {The sense of smell is vital for many animals, particularly insects, who rely on it to detect and differentiate various odours, which is critical for their survival and ability to reproduce. In insects, the antennal lobe (AL) is the main processing centre for olfactory signals, containing a complex network of neurons that analyze and integrate olfactory information.  Gap junctions are a vital element of the AL neural network and have been demonstrated to be significant in processing and transmitting olfactory information. However, their precise function within the AL system is not yet comprehensively understood. To address this knowledge gap, this study aims to use both experimental and computational methods to model the role of gap junctions within the AL circuitry. The primary focus is to create a computational model of the gap junctions that exist between local neurons and projection neurons within the AL. A biologically accurate model has been employed to mimic the behaviour of the AL neural network, with the goal of exploring the impact of gap junctions on the processing and transmission of olfactory information. The findings of our study demonstrate the crucial role that gap junctions play in synchronizing the activity of neurons within the AL circuitry. Our computational model provides a framework for comprehending the function of gap junctions within the AL network, and it could be utilized to evaluate the impact of these junctions on olfactory processing in other insects. In summary, our results contribute to a greater understanding of the functional organization of the olfactory system and may have implications for the development of insect control strategies. }
    \end{conf-abstract}
        