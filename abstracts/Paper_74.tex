
    \begin{conf-abstract}[]
        {\textbf{Illegitimate Comment Filtration Method For Social Media Applications Using Logistic Regression }}
        {\textit{Aadhithyanarayanan V A$^{1}$, Abhijith Jaideep$^{2}$, Divya K S$^{3}$}}
        {$^{1}$Adi Shankara Institute Of Engineering And Technology $\bullet$ $^{2}$Adi Shankara Institute Of Engineering  \& Technology $\bullet$ $^{3}$Adi Shankara Institute Of Engineering  \& Technology}
        {\texttt{aadhithyaanil5@gmail.com, abhijithjaideep176@gmail.com, divya.it@adishankara.ac}}
        \indexauthors{A!Aadhithyanarayanan V, Jaideep!Abhijith, S!Divya K}
        {Comment Filtration is a critical natural language processing technique that aims to deal with harmful and offensive comments on online platforms. The technique uses machine learning algorithms to classify comments as illegal or not. It builds on pre-annotated commentary material by analyzing various features such as specific words and phrases and using context analysis. Algorithms identify patterns and correlations between text and toxicity, and a model is trained to recognize these patterns. Using specific words and phrases helps algorithms identify commonly used toxic words such as profanity, hate speech and bullying. Sentiment analysis is critical in identifying comments that may not contain toxic words but are still harmful to the community. Contextual analysis helps determine whether a comment is relevant or extraneous to the discussion and is also an important factor in identifying malicious comments. Implementation of this technique can have a significant impact on moderating and maintaining online platforms and communities, curbing toxic language, promoting a positive and respectful environment, and increasing inclusion and participation. This technique offers a solution to the growing problem of toxic comments by making online platforms safer and more enjoyable for everyone.}
    \end{conf-abstract}
        