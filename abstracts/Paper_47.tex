
    \begin{conf-abstract}[]
        {\textbf{A 6G-enabled edge-assisted Internet of Drone Things ecosystem for fire detection}}
        {\textit{Amartya Mukherjee$^{1}$, Subroto Rakshit$^{2}$, Ayan Kumar Panja$^{3}$, Debashis  De$^{4}$, Nilanjan Dey$^{5}$}}
        {$^{1}$Maulana abul kalam azad university of technology $\bullet$ $^{2}$Institute of Engineering \& Management $\bullet$ $^{3}$Institute of Engineering \& Management $\bullet$ $^{4}$Maulana abul kalam azad university of technology $\bullet$ $^{5}$TINT}
        {\texttt{mamartyacse1@gmail.com, rakshitsubroto123@gmail.com, ayan.panja@iem.edu.in, debashis.de@makautwb.ac.in, nilanjan.dey@tint.edu.in}}
        \indexauthors{Mukherjee!Amartya, Rakshit!Subroto, Panja!Ayan Kumar, De!Debashis , Dey!Nilanjan}
        {Disaster detection using unmanned aerial vehicles (UAVs) is a popular study topic. The modern Internet of Drone Things (IoDT) makes use of cutting-edge, low-latency vehicle communication and edge-enabled intelligent computing concepts to support the ecosystem's advancement to the next level. Using an edge-enabled intelligent UAV network and the opportunistic MQTT protocol, we propose an ecosystem for fire detection in a smart city and smart forest scenario. We then use an intelligent deep learning model to perform precise fire detection. According to the experimental findings, an ultralow latency sparse network environment with opportunistic flooding and forwarding protocols with RTSP support can achieve a maximum message delivery ratio of 0.9. On drones or edge-enabling devices, a deep CNN model is applied. The prediction of the fire intensity that can be detected in the chosen area of interest by using EfficientNet is 99.1\% accurate.}
    \end{conf-abstract}
        