
    \begin{conf-abstract}[]
        {\textbf{Deep Learning Based Ensemble Model for Detection of Myocardial Infarction from ECG data}}
        {\textit{Dipanwita Saha$^{1}$}}
        {$^{1}$Academy of Technology}
        {\texttt{dipanwitasaha.cse@gmail.com}}
        \indexauthors{Saha!Dipanwita}
        {The myocardial infarction(MI) is the deadly state of heart attack bearing probable risk of death. The early detection of such state can easily save the human life. The ECG or electrocardiogram can be the good mean of detection of myocardial infarc-tion of any cardiac patient. The interpretation of ECG waveform can be done by the deep learning based Convolution Neural Network (CNN) model. In the present study, the accuracy of such work has been enhanced by appointing ensemble model consisting of the CNN with Random forest (RF). The extraction of features from the ECG has been taken place at CNN layer followed by myocardial detection at the  RF layer. The PCA has been used to estimate feature reduction followed by encoding of labeled multi modal data. Signal of ECG data are measured through visibly fluctuation of HRV compared to the adjacent beats. The hyper parameters have been evaluated before processing of the whole data in CNN resulting into considerable performance gain. The result obtained has got better score over the other conventional classifiers like ANN and SVM with the estimated accuracy value of 97\%.  Rhythm measurement has also been done through SDNN, RMSSD and prediction of results were evaluated through the metrics of Precision, Recall, accuracy, ROC curve in order to detect risk factors in early phase of time with the consequences of reduction in  the death rate. }
    \end{conf-abstract}
        