
    \begin{conf-abstract}[]
        {\textbf{Heart sound classification using deep-learning neural networks}}
        {\textit{Aparna P M$^{1}$, Jayalaxmi G Naragund$^{2}$, Vishwanath P Baligar$^{3}$}}
        {$^{1}$KLE Technological University $\bullet$ $^{2}$KLE Technological University, Hubballi, Karnatak $\bullet$ $^{3}$KLE Technological University}
        {\texttt{appsaparnapm@gmail.com, jayagn@kletech.ac.in, vpbaligar@kletech.ac.in}}
        \indexauthors{M!Aparna P, Naragund!Jayalaxmi G, Baligar!Vishwanath P}
        {Cardio Vascular Diseases (CVD) are a cause of concern as the statistics regarding it's mortality globally is alarming. World Health Organisation (WHO) describes CVDs as a group of disorders of the heart and blood vessels and include coronary heart disease, cerebrovascular disease, rheumatic heart disease and other conditions. Early detection of signs and symptoms of the diseases can lead to effective treatment of the diseases and  in turn prevent premature deaths. Heart sounds, brief and transient, produced due to closing of valves can be effectively made use of to recognise abnormal heart conditions. In this work, various papers have been surveyed and their accuracy rates have been considered. In an attempt to better the accuracy the authors of this paper have considered the dataset collected from Kaggle and developed a Convolutional Neural Network (CNN model) and LSTM model that classifies the heart sounds into 3 groups (classes) namely, normal, murmur and artifact. The features are extracted using Mel-Frequency Cepstral Coefficients (MFCC) Murmur class of sounds indicate a underlying serious issue. The model considered is able to achieve 96.5\\% accuracy for the test dataset. The work can be improved by adding various other layers to the model and can be deployed for use in hospitals wherein timely detection can save human life. }
    \end{conf-abstract}
        