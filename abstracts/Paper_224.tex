
    \begin{conf-abstract}[]
        {\textbf{GAN based Image Dehazing On Raspberry PI}}
        {\textit{Asfak Ali$^{1}$, Md Sohel Akhtar$^{2}$, Sheli Sinha Chaudhuri$^{3}$}}
        {$^{1}$Jadavpur University $\bullet$ $^{2}$Jadavpur University $\bullet$ $^{3}$JADAVPUR UNIVERSITY}
        {\texttt{asfakali.etce@gmail.com, mdsohelakhtar0@gmail.com, shelism@rediffmail.com}}
        \indexauthors{Ali!Asfak, Akhtar!Md Sohel, Chaudhuri!Sheli Sinha}
        {Dehazing is a critical task in computer vision, aiming to eliminate atmospheric haze from images. In recent times, Generative Adversarial Networks (GANs) have gained significant attention as a potent solution for image restoration, including dehazing. This study proposes a GAN-based dehazing model specifically designed for atmospheric haze removal in images. The model incorporates generator and discriminator components to enhance the dehazing process through adversarial training. To effectively capture spatial and contextual information, various architectures such as MANet, PSPNet and FPN are explored. Additionally, the model incorporates a Vision Transformer as an encoder block to improve feature extraction. The proposed approach is evaluated and trained on the RESIDE dataset and benchmark datasets, utilizing objective metrics like PSNR and SSIM. Furthermore, the model is implemented for real-time dehazing on a Raspberry Pi device. The experimental outcomes and comparative analyses provide compelling evidence supporting the effectiveness of the proposed approach in achieving high-quality dehazing results.}
    \end{conf-abstract}
        