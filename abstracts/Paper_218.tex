
    \begin{conf-abstract}[]
        {\textbf{Attackers have some prior beliefs: Understanding cognitive factors of confirmation bias on adversarial decisions}}
        {\textit{Harsh Katakwar$^{1}$, Cleotilde Gonzalez$^{2}$, Varun Dutt$^{3}$}}
        {$^{1}$Indian Institute of Technology Mandi $\bullet$ $^{2}$Carnegie Mellon University $\bullet$ $^{3}$IIT, Mandi}
        {\texttt{katakwarharsh@gmail.com, coty@cmu.edu, varun@iitmandi.ac.in}}
        \indexauthors{Katakwar!Harsh, Gonzalez!Cleotilde, Dutt!Varun}
        {Cyberattacks are hazardous, and honeypot deception has been shown to be successful in combating them. Due to the involvement of multiple factors in cyber situations, the adversary is likely to suffer from various cognitive biases. Confirmation bias is one of the many cognitive biases that affect adversarial decisions in cyberspace. However, little is known about the cognitive mechanisms that drive confirmation bias in adversarial decision-making. To test for confirmation bias, one hundred and twenty participants were recruited via a crowdsourcing website and were randomly assigned to one of two between-subjects conditions in a deception-based cybersecurity simulation. Results revealed the presence of confirmation bias in adversarial decisions. Thereafter, a cognitive Instance-based Learning model was built involving recency, frequency, and cognitive noise to understand the reasons behind the reliance on confirmation bias. Results revealed that participants showed reliance on recent events and high cognitive noise in their decisions. We highlight the implications of our findings for cyber decisions in the presence of deception in the real world.}
    \end{conf-abstract}
        