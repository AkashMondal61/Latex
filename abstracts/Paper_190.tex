
    \begin{conf-abstract}[]
        {\textbf{Enhancing the Cryptography Security of Message Communication by Modified Secure IDEA Algorithm}}
        {\textit{Bilas Haldar$^{1}$}}
        {$^{1}$The Neotia University}
        {\texttt{bilasphd2020@gmail.com}}
        \indexauthors{Haldar!Bilas}
        {The security of data has become a top priority in today's world. An insecure communication network makes data transmission even more challenging. Cryptography can be an effective tool to secure information and ensure the security and confidentiality of information. In cryptography encryption and decryption methods determine communication security, with key sharing. The main purpose of key exchange methods is to provide protection for technology and are not dependent on communication channel security. Encryption and decryption methods determine the transaction security with key sharing. The present work is a new method for generating keys for multi-party communications by using a quasi-group approach. An algorithm for generating a master key from an unordered value gap was proposed in this work using a quasi-group of order n. It is proposed the algorithms to generate and renew multiple subkeys from the primary key and vice versa. Additionally, the proposed methods are based on a Modified Secure IDEA (MS IDEA) algorithm for encryption and decryption using a 256-bit key.  The results showed that these methods improve in time complexity and protection against attacks.}
    \end{conf-abstract}
        