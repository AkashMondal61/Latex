
    \begin{conf-abstract}[]
        {\textbf{Modified Karatsuba Approximate Multiplier for error-resilient applications}}
        {\textit{Pathallapalli Hareesha$^{1}$, Kishore  Kumar Gundugonti$^{2}$}}
        {$^{1}$Velagapudi Ramakrishna Siddhartha Engineering College $\bullet$ $^{2}$V.R.Siddhartha Engineering College}
        {\texttt{hareeshapathallapalli08@gmail.com, gkishorekumar@vrsiddhartha.ac.in}}
        \indexauthors{Hareesha!Pathallapalli, Gundugonti!Kishore  Kumar}
        {Multiplication, being a frequently used arithmetic operation in various applications, has sparked interest in approximate computing. This approach al-lows for trading precision for benefits such as reduced hardware usage, power consumption, and delay periods. In the realm of error-tolerant applications, approximate computing has gained significant attention and requires optimization to meet system needs. The Rounding-based approximate multi-plier employing modified Karatsuba serves as the foundation for an algorithm that achieves approximate multiplication while balancing error and de-sign metrics. However, we have introduced a more efficient approach known as the OR-based approximate multiplier (OR-AM), surpassing the limitations of the rounding-based approximate multiplier. The OR-AM demonstrates superior performance by leveraging the utilization of OR gates. These gates provide several advantages, including reduced power consumption, shorter delay periods, and smaller circuit area. By incorporating OR gates strategically, the OR-AM enhances design metrics while maintaining a satisfactory lev-el of error tolerance. In image processing and other applications reliant on arithmetic operations, multipliers play a vital role. The OR-AM outperforms its predecessors by significantly reducing the number of multipliers required. Furthermore, it optimizes time, space, and power usage by utilizing small multipliers alongside shifting and rounding procedures. In summary, the OR-based approximate multiplier represents a significant advancement in approximate multiplication. By harnessing the power of OR gates, it achieves improved design metrics, surpassing the limitations of the rounding-based approach. The OR-AM's reduced hardware usage, lower power consumption, and enhanced efficiency make it a promising solution for a wide range of applications, including image processing.}
    \end{conf-abstract}
        