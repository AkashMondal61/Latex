
    \begin{conf-abstract}[]
        {\textbf{Identification And Detection of Rice Plant Diseases By Using Neural Network}}
        {\textit{RATNESH KUMAR DUBEY$^{1}$}}
        {$^{1}$IIIT Bhagalpur}
        {\texttt{ratneshnitr@gmail.com}}
        \indexauthors{DUBEY!RATNESH KUMAR}
        {The most serious disease that can affect paddy plants is blast disease. All over the world, it results in enormous yield losses. A fungus that attacks the plant's leaves, nodes, and grains is the main culprit behind the disease. Fungicides are frequently used to stop blast disease, but this approach has some drawbacks, including environmental pollution and the development of fungicide-resistant disease strains. An effective tool for modeling and managing complex systems, such as the blast disease in paddy plants, is fuzzy logic. In this study, we investigate the modeling and management of blast disease in paddy plants using fuzzy logic. We'll discuss the input variables, fuzzy sets, rule base, and output variables, among other components, that make up the fuzzy logic system. The various phases of fuzzy logic, including fuzzification, inference, and defuzzification, will also be covered. The advantages of using fuzzy logic to manage blast disease in paddy plants, including its capacity to deal with ambiguous and imprecise data and its potential to integrate with other control systems, will be discussed in the final section.The fungal disease known as rice leaf blast, which is having a devastating effect on rice production and quality throughout the globe, thrives in warm, humid environments. Management of rice production relies on precise and non-destructive diagnostic techniques. The use of hyperspectral imaging technologies for diagnosing plant diseases has much promise. The problem with using hyperspectral data to build an effective illness classification model is that it contains a lot of duplicated information. However, a lack of representative features has been gathered due to the complexity and limited scope of agricultural hyperspectral imaging data collection.This paper discussed the four models DenseNet169-MLP,CNN, EfficientNetB3,and DNN JOA. DenseNet169-MLP achieve the highest accuracy   96.52, precision of 100, F1-score 94.29 compare to other model.}
    \end{conf-abstract}
        