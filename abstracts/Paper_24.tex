
    \begin{conf-abstract}[]
        {\textbf{An Optimal Edge Server Placement Algorithm based on Glow-Worm Swarm Optimization Technique}}
        {\textit{Moumita Roy$^{1}$, Ujjwal Maulik $^{2}$, Chandreyee Chowdhury$^{3}$, Mohana$^{4}$}}
        {$^{1}$Institute of Engineering and Management $\bullet$ $^{2}$JU $\bullet$ $^{3}$Associate Professor, CSE, Jadavpur University $\bullet$ $^{4}$Kendriya Vidalaya Burdwan}
        {\texttt{moumita1055@gmail.com, ujjwalmaulik@yahoo.com, chandreyee.chowdhury@gmail.com,  mohanaprofile@gmail.com}}
        \indexauthors{Roy!Moumita, !Ujjwal Maulik, Chowdhury!Chandreyee, Mohana!}
        {Mobile edge computing is nowadays an emerging and prospective computing paradigm to encompass a further dimension of ubiquitous computing. Here, edge servers are deployed in close proximity to mobile devices to promote low latency and improve the present network architecture. Computation and storage are transferred from the core network to the edge network for faster and more reliable service. However, the primary challenge in the deployment of mobile edge computing architecture is the optimal placement of the edge servers since finding an appropriate location for the edge servers is fundamental and critical. In this paper, we have formulated the edge server placement problem as a constraint optimization problem that places edge servers strategically to balance their workloads and reduce access delay between the base stations and edge servers. A modified GSO algorithm is proposed to find the optimal solution that minimizes the number of edge server requirements. The proposed approach is simulated in MATLAB. The results demonstrate the behavior of the proposed approach subject to varying densities of base stations with respect to clusters. The proposed algorithm is also found to perform better in comparison to state-of-the-art Edge Server Placement algorithms.}
    \end{conf-abstract}
        