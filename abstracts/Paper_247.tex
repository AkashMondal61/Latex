
    \begin{conf-abstract}[]
        {\textbf{Leveraging POS-tag features for Machine Translation of the Bengali-Nepali Language Pair: A Preliminary Study}}
        {\textit{Pooja  Rai$^{1}$}}
        {$^{1}$Indian Institute of Information technology, Kalyani}
        {\texttt{poojaphd@iiitkalyani.ac.in}}
        \indexauthors{Rai!Pooja }
        {Machine Translation (MT) plays a crucial role in break- ing language barriers and facilitating cross-lingual communication. How- ever, certain language pairs, particularly those with limited linguistic resources, pose significant challenges in building effective MT systems. In this paper, we present the first attempt to build a machine transla- tion system for the Bengali-Nepali language pair, which lacks an existing MT solution. To address the data scarcity issue, we utilized the Bengali- Nepali parallel treebanks, as the training resources for both statistical machine translation (SMT) and neural machine translation (NMT) ap- proaches. We adopted Moses, a widely-used SMT framework, to develop an initial baseline system. Additionally, we integrated a factored lan- guage model that incorporates Part-of-Speech (POS) features to enhance the SMT-based translation performance. To further explore the potential of NMT for this low-resource language pair, we also experimented with transformer-based architecture. We leveraged the POS information to augment the transformer model and improve its translation capabilities. Our findings reveal that the inclusion of POS features in both SMT and NMT models leads to noticeable enhancements in translation quality. By building upon our initial findings, future research could potentially ad- dress the challenges posed by the scarcity of parallel data and contribute to more effective and reliable MT solutions for this specific language pair.}
    \end{conf-abstract}
        