
    \begin{conf-abstract}[]
        {\textbf{Comparative Study For Fraud Detection In Credit Card Transaction Data Using Machine Learning Algorithms}}
        {\textit{Aditi Biswas$^{1}$}}
        {$^{1}$Amity University Kolkata}
        {\texttt{aditibiswas002@gmail.com}}
        \indexauthors{Biswas!Aditi}
        {In the card-not-present space it is easier for fraudsters to defraud customers and much more riskier for the merchants since the fraudster is anonymous in such scenarios.This paper proposes to combine the individual predictions from Double Q Learning,Support Vector Machine and Random Forest models using their weighted average to inform the development of more effective fraud prevention strategies.To validate the effectiveness of the proposed model, different evaluation metrics such as accuracy, precision, AUROC and F1-Score have been considered.Random Forest Model performed best in terms of all the performance metrics followed by our ensemble model. This study aims to provide researchers with valuable insights for selecting an accurate model to classify fraudulent transactions, especially when dealing with imbalanced datasets. The findings of this research will empower fraud analysts and professionals, allowing them to elevate their efficiency and precision in classifying fraudulent transactions.}
    \end{conf-abstract}
        