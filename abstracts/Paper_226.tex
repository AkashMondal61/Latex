
    \begin{conf-abstract}[]
        {\textbf{Privacy Preservation of Insurance Data Sharing Across Permissioned Blockchains}}
        {\textit{Chappidi Harika$^{1}$, Susmita Mandal$^{2}$}}
        {$^{1}$IDRBT $\bullet$ $^{2}$IDRBT}
        {\texttt{charika@idrbt.ac.in, msusmita@idrbt.ac.in}}
        \indexauthors{Harika!Chappidi, Mandal!Susmita}
        {With the rise of blockchain as a more promising technology, industries and academia are interested in foreseeing a decentralized and transparent ecosystem of data storing, sharing, and verifying across platforms. The current state-of-the-art blockchain platforms mostly work in isolation with a high level of diversity due to architecture, security, and efficiency. Thus making interoperability difficult. An optimistic approach can be through cross-chain technology that has the ability to transfer data and assets between different blockchains. Recently, security concerns related to cross-chain data exchange have gained a lot of attention. This paper is a work at two folds; first, we analyze and test the issue of oversharing of sensitive data across permissioned platforms, such as Hyperledger Fabric (HLF). Second, a smart contract-driven novel approach of authorizing the insurance claimant while preserving sensitive information is designed using NIZKP. This prevents unauthorized data sharing or misuse in the ecosystem. Finally, the results of the solution are presented with security analysis.}
    \end{conf-abstract}
        