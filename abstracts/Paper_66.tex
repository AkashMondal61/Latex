
    \begin{conf-abstract}[]
        {\textbf{Detection of Diseases in Mango Leaves Based on Different Classification Algorithms and Their Comparisons Using Python}}
        {\textit{Pintu Das$^{1}$}}
        {$^{1}$MCKV Institute of Engineering}
        {\texttt{daspintu2001@gmail.com}}
        \indexauthors{Das!Pintu}
        {Abstract. This article presents an extended investigation regarding the identification of ambience of leaf illness. In this regard, the investigation has been carried out using 4000 mango leaf samples. On the basis of this investigation, the classification grade of unhealthy leaves and the rate of infection have been obtained through peripheral extraction and contour segmentation procedures. Support Vector Machine (SVM), Random Forest (RF), Decision Tree (DT), and Confusion Matrix (CM) models have been used for leaf disease detection. Based on this absolute classification, it automatically acquires the ability to extract very useful features. As a result of the SVM, the methodology presented is the most accurate for datasets at 97.88\%. The obtained experimental results prove that among all the algorithms used for classification, SVM is the most efficient in detecting mango leaf disease. Vision-based leaf disease presented adequate results and excellent working efficiency.  Keywords: Segmentation, Classification, Support Vector Machine, Random Forest. }
    \end{conf-abstract}
        