
    \begin{conf-abstract}[]
        {\textbf{Identification of Human Drug Targets for COVID-19 Based on Subcellular Localization Information, Gene Expression Data and Node2vec}}
        {\textit{Chandrima Das$^{1}$, Sovan Saha$^{2}$}}
        {$^{1}$Institute of Engineering and Management $\bullet$ $^{2}$Techno Main Salt Lake}
        {\texttt{chandrimapfl@gmail.com, sovansaha12@gmail.com}}
        \indexauthors{Das!Chandrima, Saha!Sovan}
        {For the last few years, humans have been searching for a good solution to the lethal disease of COVID-19. Finding human proteins that can be drugged to fight diseases like this is vital. Identification of essential proteins is an integral part of human drug design. Wet lab studies like single gene knockouts, RNA interference, and anti-sense RNA are employed to discover essential proteins. Despite being exceedingly accurate, these methods are costly and time-consuming. So, this led to an increase in the demand for computational methods. The two main categories for computational methods are topology-based methods and sequence-based methods. While sequence-based methods predict essential proteins using the sequence features of proteins, the former uses the topological aspects of protein-protein Interaction Networks. Simply using topological features to predict essential proteins produces less accurate predictions. Hence, this research proposes a machine learning-based methodology for identifying COVID-19 drug targets by integrating several biological information. RNA-Sequence and subcellular location data are used as the features. This forms the weighted PPI network using Pearson Correlation coefficient after performing Principal Component Analysis. The topological features for the proteins are obtained from the feature vector set using node2vec. Lastly, the best features are selected, which will be used as input for the ML models. Machine learning models like Naïve Bayes Classifier, Logistic Regression, Random Forest, AdaBoost, Support Vector Classification and XGBBoost Classifier are trained on the data. The best-performing models are then used to identify novel COVID-19 drug targets.}
    \end{conf-abstract}
        