
    \begin{conf-abstract}[]
        {\textbf{Hypermotifs in biological networks: TGFβ induced EMT as a case study}}
        {\textit{Sai Bhavani Gottumukkala$^{1}$, Anbumathi Palanisamy$^{2}$}}
        {$^{1}$National Institute of Technology Warangal $\bullet$ $^{2}$National Institute of Technology Warangal}
        {\texttt{gsaib@student.nitw.ac.in, anbu@nitw.ac.in}}
        \indexauthors{Gottumukkala!Sai Bhavani, Palanisamy!Anbumathi}
        {Complex systems describe a collection of entities and interactions influencing each other and are prevalent in various domains. Networks are one of the common ways to represent these complex systems across various disciplines. In biology, networks provide a mathematical constriction to interpret the association between biological molecules. Network motifs are the small subgraph patterns within the large biological networks which play several crucial functional roles. While network motifs like coherent feedforward loops are extensively analyzed, there are limited studies on how these network motifs are combined to form hypermotifs within the biological networks, and what properties may emerge from these hypermotifs are not extensively studied. Recent studies highlighted the prevailing role of hypermotifs in various contexts (Adler \& Medzhitov, 2022; Sai Bhavani \& Palanisamy, 2023). This work presents some of the hypermotis observed in regulating Epithelial Mesenchymal Transition (EMT) and cancer metastasis. Further, the systems level modeling results of these hypermotifs towards illustrating their emergent behavior using the discreet and continuous dynamic modeling approaches were presented. Understanding the emergent properties of hypermotifs contributes to our broader comprehension of network-based regulation in complex biological systems such as,  cancer metastasis. }
    \end{conf-abstract}
        