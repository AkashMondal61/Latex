
    \begin{conf-abstract}[]
        {\textbf{Assessment of Cardiac Autonomic Modulation parameters in a Healthy Population}}
        {\textit{Raghuwansh Sing$^{1}$, Vivek Ranjan$^{2}$, Anindita Ganguly$^{3}$, Suman Halder$^{4}$}}
        {$^{1}$NIT DURGAPUR $\bullet$ $^{2}$National Institute of Technology Durgapur $\bullet$ $^{3}$Director of Technical Education, Government of West Bengal $\bullet$ $^{4}$NIT Durgapur}
        {\texttt{rs.22ee1105@phd.nitdgp.ac.in, vr.21ee1107@phd.nitdgp.ac.in, dtewbgovt@gmail.com, suman.halder@ee.nitdgp.ac.in}}
        \indexauthors{Sing!Raghuwansh, Ranjan!Vivek, Ganguly!Anindita, Halder!Suman}
        {Heart rate variability (HRV), a physiological measure, can manifest changes in stress levels even when other physiological variables like blood pressure are within normal bounds. The findings of this study proclaim that the enactment of meditation has a calming effect on the nerv-ous system, as it ushers to a reduction in sympathetic activity and an enhancement in parasympa-thetic activity. The study necessitated a cohort of eight participants, comprising five females and three males. A total duration of 50 minutes of electrocardiogram (ECG) recording is conducted, encompassing both the pre-meditation and meditation periods. This study examines the significant changes in Time Domain HRV indices, including mean RR interval (Mean RR), standard devia-tion of normal-to-normal intervals (SDNN), mean heart rate (Mean HR), standard deviation of heart rate (STD HR), maximum heart rate, HRV triangular index, TINN, and stress index, with a statistical significance level of p<0.05. The results of the NL indicate that there are statistically significant changes in various measures of HRV, specifically Poincare plot SD2 (SD2), Poincare plot SD2/SD1, Approximate Entropy (ApEn), Sample Entropy (SampEn), Detrended Fluctuation Analysis (DFA alpha 1), and (DFA alpha 2), at a significance level of p<0.01. The statistical analysis conveys that there is a significant difference in the Frequency Domain Peak High Fre-quency (Peak HF), Power (ms^2) LF, Power (Log) LF, Power (\%) VLF, Power (\%) LF, Power (\%) HF, Power (nu) LF, Power (nu) HF, and the ratio of low frequency to high frequency (LF/HF) at a significance level of p < 0.01.}
    \end{conf-abstract}
        