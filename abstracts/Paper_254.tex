
    \begin{conf-abstract}[]
        {\textbf{Survey on Cyber-security for Health-care System}}
        {\textit{Md Fahin Parvej$^{1}$, Ajoy Kumar Khan$^{2}$, Dipak kole$^{3}$}}
        {$^{1}$Jalpaiguri Government Engineering College $\bullet$ $^{2}$Mizoram University $\bullet$ $^{3}$jgec}
        {\texttt{fpar570@gmail.com, mzut250@mzu.edu.in, dipak.kole@cse.jgec.ac.in}}
        \indexauthors{Parvej!Md Fahin, Khan!Ajoy Kumar, kole!Dipak}
        {Healthcare system, now-a-days ,uses a wide range of digital devices, systems and resources. For the advent of the internet ,e-health users can transfer health related data to healthcare providers without delay. It also helps to engage in person-toperson exchange of video, audio, text, and other types of data, to research health information, and to access medical records. However, without any security measures, hospitals can be easy targets for data breaches, ransomware attack, Denial-Of-Service-Attack, phishing, just to name a few. In this paper, to examine the privacy and security of e-health systems, major components of the present day wireless healthcare system have been identified. Recent security and privacy studies that focus on components of the e-health systems have been reviewed. On the basis of the review, open challenges, research trends , solutions, requirements, security issues of a wireless healthcare system have been obtained.}
    \end{conf-abstract}
        