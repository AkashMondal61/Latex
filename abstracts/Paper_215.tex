
    \begin{conf-abstract}[]
        {\textbf{Autism Spectrum Disorder Detection through Facial Analysis and Deep Learning: Leveraging Domain-Specific Variations}}
        {\textit{krishna sai koppula$^{1}$}}
        {$^{1}$IIIT Allahabad}
        {\texttt{mit2021046@iiita.ac.in}}
        \indexauthors{koppula!krishna sai}
        { Autism Spectrum Disorder (ASD) is a neurodevelopmen- tal disorder with significant implications for individuals and society. Early detection of ASD is crucial for effective intervention and sup- port.This study focuses on the detection of ASD using facial analy- sis and deep learning techniques. The primary objective is to lever- age domain-specific variations by employing four different CNN archi- tectures, including VGG16, ResNet50, SE-ResNet50, and MobileNetv2. Unlike previous research that uses the ImageNet dataset, our models are pre-trained on the specialized VGGFace2 dataset, enabling them to capture subtle facial variations and face-specific granular features.The performance of our models is compared with their counterparts based on the same architecture. The experimental results demonstrate that our models consistently outperform the existing methods in terms of various metrics, including test accuracy, AUC.Specifically, the VGG16 model achieved a test accuracy of 0.86 and an AUC of 0.86, surpass- ing the performance of other studies. Similarly, the ResNet50 and Mo- bileNetv2 models demonstrated superior performance compared to their research counterparts.These findings highlight the effectiveness of lever- aging domain-specific variations in facial analysis for ASD detection.}
    \end{conf-abstract}
        