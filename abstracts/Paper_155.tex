
    \begin{conf-abstract}[]
        {\textbf{Predicting suicidal behavior among Indian adults using childhood trauma, mental health questionnaires and machine learning cascade ensembles}}
        {\textit{Akash Rao$^{1}$, Gajraj S Chouhan$^{2}$, Varun Dutt$^{3}$}}
        {$^{1}$IIT Mandi $\bullet$ $^{2}$IIT Mandi $\bullet$ $^{3}$IIT, Mandi}
        {\texttt{akashrao.iitmandi@gmail.com, b19130@students.iitmandi.ac.in, varun@iitmandi.ac.in}}
        \indexauthors{Rao!Akash, Chouhan!Gajraj S, Dutt!Varun}
        {Among young adults, suicide is India's leading cause of death, ac-counting for an alarming national suicide rate of around 16\%. In recent years, machine learning algorithms have emerged to predict suicidal behavior using various behavioral traits. But to date, the efficacy of machine learning algorithms in predicting suicidal behavior in the Indian context has not been explored in literature. In this study, different machine learning algorithms and ensembles were developed to predict suicide behavior based on childhood trauma, different mental health parameters, and other behavioral factors. The dataset was acquired from 391 individuals from a wellness center in India. Information regarding their childhood trauma, psychological wellness, and other mental health issues was acquired through standardized questionnaires. Results revealed that cascade ensemble learning methods using support vector machine, decision trees, and random forest were able to classify suicidal behavior with an accuracy of 95.04\% using data from childhood trauma and mental health questionnaires. The study highlights the potential of using these machine learning ensembles to identify individuals with suicidal tendencies so that targeted interventions could be provided efficiently.}
    \end{conf-abstract}
        