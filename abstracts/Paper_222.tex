
    \begin{conf-abstract}[]
        {\textbf{Ternary D Flip-Flop in CNFET-Memristor Technology}}
        {\textit{Shivani Thakur$^{1}$, Srinivas Bodapati$^{2}$}}
        {$^{1}$IIT MANDI $\bullet$ $^{2}$IIT Mandi}
        {\texttt{s21006@students.iitmandi.ac.in, srinivasu@iitmandi.ac.in}}
        \indexauthors{Thakur!Shivani, Bodapati!Srinivas}
        {This paper presents the design of a ternary D flip-flop to store a ternary digit in Carbon Nanotube FET (CNFET)-Memristor technology. This paper presents three designs of the D flip-flop, one is a latch-based edge-triggered D flip-flop using master-slave configuration, named as LDFF. The other designs use a transmission gate and a buffer to design a positive edge-triggered D flip-flop. A ternary buffer is designed using a standard ternary inverter and ternary cycle operators. A cycle operator-based ternary D flip-flop is named as CDFF, which uses cycle operators in the loop to store a digit. A positive edge-triggered D flip-flop using a standard ternary inverter is named SDFF, which is having less number of CNFETs and memristors. Out of these three designs, SDFF is having better PDP compared to the other designs. In particular, the SDFF takes 40\% and 63\% of energy compared to the LDFF and CDFF. We have extended the ternary D flip-flop to design a synchronous ternary counter. Extensive simulations were performed, and simulation plots were presented. }
    \end{conf-abstract}
        