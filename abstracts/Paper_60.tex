
    \begin{conf-abstract}[]
        {\textbf{A Comparative Analysis of Feature Selection Approaches for Sensor-based Human Activity Recognition}}
        {\textit{Anindita  Saha$^{1}$, Prasanta Sen$^{2}$, Saroj  Kumari$^{3}$, Chandreyee Chowdhury$^{4}$}}
        {$^{1}$Techno main saltlake $\bullet$ $^{2}$Techno Main saltlake $\bullet$ $^{3}$Techno Main saltlake $\bullet$ $^{4 ""}$Associate Professor, CSE, Jadavpur University}
        {\texttt{aninditasaha03@yahoo.co.in, prasantasen540@gmail.com, s1028510@gmail.com, chandreyee.chowdhury@gmail.com}}
        \indexauthors{Saha!Anindita , Sen!Prasanta, Kumari!Saroj , Chowdhury!Chandreyee}
        {In the era of infrastructure-less sensing, human activity recog- nition takes a new leap by exploiting the ubiquity of smartphone sen- sors. However, the limited computational capability of the smart hand- helds hampers the inherent need for real-time responsiveness of the ap- plication. Thus, dimensionality reduction through Feature Selection (FS) techniques could be a precursor for on-device learning and prediction of activities. Most of the existing FS approaches focus on the algorithmic perspectives and ignore the data-intensive nature of the application. So, in this paper, we perform a comparative study of the FS techniques uti- lized for HAR subject to the role of feature preprocessing. Apart from de- tailing the FS methods, the work also shows how to apply those methods in combination. The experimental results across the benchmark datasets show the efficiency of our application strategies of the FS techniques.}
    \end{conf-abstract}
        