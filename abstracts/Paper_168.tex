
    \begin{conf-abstract}[]
        {\textbf{Survey on Deep Learning Technique on Maize Leaves Infected by Fall Armyworms}}
        {\textit{Soumadeep Bagui$^{1}$, Dipak kole$^{2}$, Anilabha  Datta$^{3}$, Akash Mondal$^{4}$, Avishek  Chatterjee$^{5}$, Kusal Roy$^{6}$}}
        {$^{1}$Jalpaiguri Government Engineering College $\bullet$ $^{2}$jgec $\bullet$ $^{3}$Jalpaiguri Government Engineering College $\bullet$ $^{4}$Jalpaiguri Government Engineering College $\bullet$ $^{5}$Jalpaiguri Government Engineering College $\bullet$ $^{6}$Bidhan Chandra Krishi Viswavidyalaya}
        {\texttt{sb2310@cse.jgec.ac.in, dipak.kole@cse.jgec.ac.in, ad2307@cse.jgec.ac.in, akashramnagar282@gmail.com, ac2301@cse.jgec.ac.in, roy.kusal@bckv.edu.in}}
        \indexauthors{Bagui!Soumadeep, kole!Dipak, Datta!Anilabha , Mondal!Akash, Chatterjee!Avishek , Roy!Kusal}
        {One of the most commercially successful cereal crops farmed worldwide is maize (Zea mays L.), which is a common staple food in many developing nations. Every year we encounter massive yield losses due to infestation of fall armyworms. The goal of this work is to develop a deep learning-based model that is trained on images of healthy and fall armyworm-infested maize leaves from a dataset using Convolutional Neural Network (CNN) architecture and enhance the accuracy using Ensemble learning. So, classification of leaves on the basis of infestation of fall armyworms , that is healthy or infected , can lead to early detection of infestation in leaves and steps can be taken as per requirement.}
    \end{conf-abstract}
        