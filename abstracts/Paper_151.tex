
    \begin{conf-abstract}[]
        {\textbf{Quantum Resistant Hash-Based Digital Signature Schemes: A Review}}
        {\textit{Swarna Panthi$^{1}$}}
        {$^{1}$North eastern hill university}
        {\texttt{swarnapanthi1997@gmail.com}}
        \indexauthors{Panthi!Swarna}
        {Our trust in traditional digital signature schemes has been shaken by the current development towards quantum computers. Digital signatures are currently based on public key cryptosystems like El- Gamal, Elliptic Curve Digital Signature Algorithm (ECDSA), Rivest, Shamir, Adleman algorithm (RSA), etc. The security of these public key cryptosystems is based on the assumptions of the computational infeasibility of a few known mathematical problems and several quantum algorithms such as Shor etc. exist for solving these mathematical problems. Therefore a lot of research is ongoing on the development of new signature schemes called Post-quantum digital signatures that can be used on both classical and quantum computers. The major families of postquantum signatures are multivariate, isogeny-based, lattice-based, noncommutative, code-based, and hash-based. This paper focuses on hash-based signature schemes. Hash-based signature schemes' security has been firmly established against both classical and quantum attacks, making them a secure, efficient, and prominent candidate for post-quantum digital signature schemes. The objective of this paper is to provide an overview of various hash-based signature schemes and contrast them under different criteria. This paper also presents the implementation results of the schemes to understand the working and correlations among different parameters which will help future researchers to develop signature schemes for different environments.}
    \end{conf-abstract}
        