
    \begin{conf-abstract}[]
        {\textbf{Unveiling the art of music generation with LSTM}}
        {\textit{Nishtha  Srivastava$^{1}$, Sankita J. Patel$^{2}$}}
        {$^{1}$NIT,SURAT $\bullet$ $^{2}$NIT Surat}
        {\texttt{d20co005@coed.svnit.ac.in, sjp@coed.svnit.ac.in}}
        \indexauthors{Srivastava!Nishtha , Patel!Sankita J.}
        {Music plays a significant role in various aspects of human life. With the progress of technology, music has evolved, and artificial intelligence has become a valuable tool for newcomers in the creative process. However, manually or automatically generating music is a complex task due to the many variables involved. Different approaches leverage deep learning to generate music using different types of musical data, such as raw waveforms or structured formats like MIDI or MusicXML. In this research, sequence-based models like LSTM are employed for this purpose, demonstrating their effectiveness in capturing sequence information. The goal is to artificially create human-sounding music, adhering to established rules and classic rhythms while fostering originality. This paper introduces an LSTM network for music generation, providing an accessible overview of the underlying theory and its application in modeling music sequences. Additionally, an analysis is conducted to evaluate the advantages and disadvantages of the proposed model. The qualitative impact of the model output is assessed, and potential areas for improvement are discussed, guiding future work in this field.}
    \end{conf-abstract}
        