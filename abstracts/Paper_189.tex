
    \begin{conf-abstract}[]
        {\textbf{Supervised Machine Learning Algorithms for Malware Classification - A Comparative Analysis}}
        {\textit{Anisha Mahato$^{1}$, Dr. R.T Goswami$^{2}$, Ambar Dutta$^{3}$}}
        {$^{1}$Techno International New Town $\bullet$ $^{2}$Techno India College of Technology Rajarhat $\bullet$ $^{3}$Amity University, Kolkata}
        {\texttt{anisha.mahato@tint.edu.in, rtgoswami@tict.edu.in, ambardutta@gmail.com}}
        \indexauthors{Mahato!Anisha, Goswami!Dr. R.T, Dutta!Ambar}
        {With the ever-increasing reliance on digital technologies and the pervasive nature of the internet, the threat posed by malware has become a significant concern in today's interconnected world. The primary objective of malware classification is categorization malware samples into distinct families or classes based on their behavior, characteristics and intent. Traditional signature-based detection methods, though effective against known malware, struggle to identify previously unseen or polymorphic variants. Hence the need for more sophisticated approaches that employ machine learning occurred. In this paper, a benchmarked dataset Big 2015 is used for the malware classification experiment. Seven different machine learning models namely Random Forest, Support Vector Machines, Logistic Regression, Naïve Bayes, Ada Boost, Gradient Boost and Bagging are used to train and test the dataset and to establish the one that performs the best. The performance metrices for evaluation are Accuracy, F1-score, Precision and Recall. It is seen that ensemble machine learning approach namely Random Forest, Bagging and Gradient Boost performed better in accordance to the performance parameters considered.}
    \end{conf-abstract}
        