
    \begin{conf-abstract}[]
        {\textbf{Few shot learning with fine-tuned language model for suicidal text detection}}
        {\textit{Shivam Shivam$^{1}$, Biswarup Ray$^{2}$}}
        {$^{1}$ZS Associates $\bullet$ $^{2}$ZS}
        {\texttt{shivam.shivam@zs.com, raybiswarup9@gmail.com}}
        \indexauthors{Shivam!Shivam, Ray!Biswarup}
        {The most effective way to prevent potential suicide attempts is early identification followed by prompt treatment. Nowadays, online communication channels especially social media is becoming a way of expressing suicidal tendencies. However, the availability of properly tagged textual data is a major issue for researchers to perform identification or classification of such tendencies through texts. In light of the abovementioned facts, in the present work, we have proposed an approach to determine early detection with the goal of early diagnosis of suicidal behaviour through text posted on social media via supervised learning using few shot learning process. Therefore, to detect suicidal behaviour, we extract embeddings by finetuning a large pretrained language model. The contextual embeddings have then been used for binary classification of the text for suicidal behaviour, in which a comparison of various classifiers has been performed. This includes traditional supervised classifiers and neural network models. A comparison of the model's performance with and without an outlier detection and removal step has also been performed to highlight the importance of the outlier detection step in the pipeline. The feasibility and practicality of the approach have been demonstrated by generating results for user-generated content scraped from Reddit platform posts made on subreddits 'SuicideWatch' and 'depression'. The results generated by the model are also seen to outperform various state-of-the-art models.}
    \end{conf-abstract}
        