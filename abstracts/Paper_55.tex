
    \begin{conf-abstract}[]
        {\textbf{Potential interaction of vitamins with different signaling pathways to inhibit the growth of T-cell lymphoma}}
        {\textit{Tunnisha Dasgupta$^{1}$, Soham Sen$^{2}$, Moumita Mondal$^{3}$, Sikta Mondal$^{4}$, Tanumoy Banerjee$^{5}$, Ujjayan Majumdar$^{6}$}}
        {$^{1}$Jalpaiguri Government Engineering College $\bullet$ $^{2}$Jalpaiguri Government Engineering College $\bullet$ $^{3}$Jalpaiguri Government Engineering College $\bullet$ $^{4}$Jalpaiguri Government Engineering College $\bullet$ $^{5}$Lehigh University $\bullet$ $^{6}$The State University of New York at Buffalo}
        {\texttt{td2428@ce.jgec.ac.in, ss2412@ee.jgec.ac.in, mm2423@ce.jgec.ac.in, sm2535@it.jgec.ac.in, tab220@lehigh.edu, ujjayanm@buffalo.edu}}
        \indexauthors{Dasgupta!Tunnisha, Sen!Soham, Mondal!Moumita, Mondal!Sikta, Banerjee!Tanumoy, Majumdar!Ujjayan}
        {           T-cell lymphomas account for around 5-10\% of all NHLs in the Western Hemisphere and 15-20\% of all NHLs in Asian countries. One of the significant advantages of treatment with vitamin-related derivatives is the lack of side effects. It has been observed in recent studies that vitamins have shown immense potential treatment to inhibit the growth of cancer cells. However, studies regarding the mechanisms to inhibit the growth of cancer cells are still far from reality. Several gene expressions are involved in T-cell lymphomas' growth, including Wnt/β-catenin, NF-κB, and notch1, which assist by signaling. These signaling pathways are often responsible for the growth of T-cell lymphoma. In this study, molecular docking was performed to investigate the interaction of different vitamins with these signaling gene expressions/ proteins. Vitamin A shows the highest binding affinity as compared to other vitamins. On the other hand, β-catenin shows the highest binding affinity compared to other proteins with the vitamins. Whether this binding will lead to upregulation or downregulation of the gene expressions is yet to be studied, and further experimentation is required, such as RT-qPCR. However, the significant binding affinity can help in the identification of drugs that can interact with the gene expression and assist in its downregulation, which will initiate the inhibition of lymphoma cell growth. }
    \end{conf-abstract}
        