
    \begin{conf-abstract}[]
        {\textbf{Enhancing Driver Safety and Experience: Real-Time Facial Expression Detection in Smart Vehicles with Vision Systems}}
        {\textit{Dr Soumya Ranjan Mishra$^{1}$, Hitesh Dr Mohapatra$^{2}$}}
        {$^{1}$KIIT University $\bullet$ $^{2}$KIIT Deemed to be University}
        {\texttt{soumyaranjan.mishrafcs@kiit.ac.in, hiteshmahapatra.fcs@kiit.ac.in}}
        \indexauthors{Mishra!Dr Soumya Ranjan, Mohapatra!Hitesh Dr}
        {One of the primary causes of mortality nowadays is traffic accidents. several of these accidents are brought on by drowsy driving, which has several detrimental implications. There are a few techniques proposed by different researchers like lane departure warning: This technique uses a camera to detect lane markings on the road and warns the driver if the vehicle drifts out of the lane without signalling. This can help prevent accidents caused by distracted or drowsy driving. Forward collision warning: This technique uses sensors to detect the distance between the vehicle and objects in front of it, and informs the driver if the vehicle is approaching an item or another vehicle too closely. By doing this, rear-end accidents may be avoided. In this study, we suggested a system that analyses a driver's facial expression to conclude whether or not the driver is drowsy. This image-based system monitors the driver's alertness and detects sleepiness by employing convolutional neural networks for computer vision and machine learning. To determine whether a person is feeling sleepy, this technology analyses facial characteristics.}
    \end{conf-abstract}
        