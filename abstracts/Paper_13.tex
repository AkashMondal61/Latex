
    \begin{conf-abstract}[]
        {\textbf{A Multilayer Framework for Data Driven Student Modeling}}
        {\textit{Mitra  Datta$^{1}$, Dr Sobin CC$^{2}$}}
        {$^{1}$SRM University $\bullet$ $^{2}$SRM University, Amaravati, AP}
        {\texttt{mitradattag@srmap.edu.in, sobin.c@srmap.edu.in}}
        \indexauthors{Datta!Mitra , CC!Dr Sobin}
        {The significance of student affective states in the learning process has been widely acknowledged. These emotional states experienced by students play a crucial role in shaping their engagement, motivation, and overall learning experience. Simultaneously, studies in learning analytics have showcased the potential of leveraging the abundant data gathered by e-learning systems such as ITS to identify patterns in student behavior for highlighting the indicators of student learning outcomes. In our current research investigation, we have examined four crucial affective states, namely Boredom, Frustration, Confusion, and Concentration, along with student interaction data collected from the Assistments platform. Our research has a dual purpose. The first objective is to gain insights into the influence of student affective states and scaffolding on learning progress within a blended learning environment. The second objective is to develop a transparent,data-driven framework based on machine learning that improves educational outcomes. We utilized clickstream and knowledge component data to employ four machine learning models, specifically RandomForest, Linear Regression and neural network, for the prediction of student performance. We compared these models bothin the traditional approach and our proposed framework, MLF-DSM.. Furthermore,we employed XAI techniques, specifically SHAP to enhance the transparency of the results obtained from this black box model. Overall, our study highlights the applicability of MLF-DSM as a dynamic assessment and personalization tool within a blended learning environment.}
    \end{conf-abstract}
        