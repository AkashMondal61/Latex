
    \begin{conf-abstract}[]
        {\textbf{Rough Enhanced Fuzzy Segmentation Algorithm for Region Detection}}
        {\textit{Prattay Paul$^{1}$, Soham Ghosh$^{2}$, Amiya Halder$^{3}$}}
        {$^{1}$St. Thomas' College of Engineering and Technology $\bullet$ $^{2}$St. Thomas' College of Engineering and Technology $\bullet$ $^{3}$St Thomas' College of Engineering and Technology}
        {\texttt{prattaypaul2@gmail.com, ghoshsoham1982@gmail.com, amiya.halder77@gmail.com}}
        \indexauthors{Paul!Prattay, Ghosh!Soham, Halder!Amiya}
        {One of the most common types of anomalies that happens to humans is skin diseases. They may be caused by different viruses, fungal infections, etc. Rubella is one of the most common diseases, caused by the Rubella virus. Image processing techniques play a crucial role in helping identify and locate infected regions, simplifying the process of diagnosing and treating ailments. A new proposed technique is developed as an attempt to identify the Rubella virus-infected areas in any part of the body. The method uses an image of the body part as input, and the algorithm can detect virus-affected regions. This is achieved by segmentation using the proposed Rough Enhanced Fuzzy C-Means (RENFCM) algorithm and binarization of segmented image, anteceded by some pre-processing techniques.}
    \end{conf-abstract}
        