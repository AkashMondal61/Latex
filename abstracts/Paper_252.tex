
    \begin{conf-abstract}[]
        {\textbf{Exploration of Graphene as Emerging 2D Material and its Applications: A Review}}
        {\textit{Malvika$^{1}$}}
        {$^{1}$National Institute of Technology Silchar}
        {\texttt{malvikathakur16@gmail.com}}
        \indexauthors{Malvika!}
        {Recently two-dimensional materials got lot of interest due to its exceptional properties. One of them is monolayer Graphene arranged in a planar sheet with sp2-bonded carbon atoms. These properties include remarkable mechanical strength, large specific surface area, excellent electronic and thermal conductivity, and other unique properties have opened up a wide range of applications for graphene. The synthesis of high-quality graphene on a large scale has been a subject of intensive research, and several techniques have been developed for this purpose. This review brings the broad canvass of the ongoing state of graphene as an effective reinforcement in various fields. Different forms of graphene such as single layer (SLG), bi-layer (BLG) and few layer graphene (FLG) has been discussed with its structure and the comparison of their numerous properties is presented in tabular form. Further, various methods used for the preparation of graphene, followed by an exploration of its structural and material properties. Moreover, the review explores the exceptional evolution of graphene in various fields. The unique properties and potential applications make graphene a promising contender for revolutionizing the semiconductor industries. As research and development efforts continue, graphene is poised to play a pivotal role in shaping the future of electronics and related industries. Finally, the article is summarized by portraying the fabrication and application difficulties of graphene and various future prospects to circumvents these issues in future.}
    \end{conf-abstract}
        