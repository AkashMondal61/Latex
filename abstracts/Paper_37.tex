
    \begin{conf-abstract}[]
        {\textbf{Prediction of S-Palmitoylation sites in the  Male/Female Mouse using Protein Language Model}}
        {\textit{Tapas Chakraborty$^{1}$, Subhadip Basu$^{2}$}}
        {$^{1}$Jadavpur University $\bullet$ $^{2}$Jadavpur University}
        {\texttt{ju.tapas@gmail.com, subhadip.basu@jadavpuruniversity.in}}
        \indexauthors{Chakraborty!Tapas, Basu!Subhadip}
        {S-palmitoylation, i.e. Post-translational modification of cysteine thiol side chain, is crucial in various biological processes and human diseases. Therefore, identification of S-palmitoylation sites from protein sequences is important, especially for understanding their functional consequences. Deep language models have shown impressive results in natural language tasks. They have recently been repurposed to biological sequences as well. Major objective of this paper is to examine whether deep language models can identify S-palmitoylation cites more efficiently. Three categories of synaptic protein datasets were considered for this experiment: male mouse, female mouse, and combination of both. Weighted data samples from each group was used for training while held-out data was used for testing and performance comparison purpose. Proposed method performed much better than the state-of-the-art approaches. Accuracy improvements on the hold-out data set are male—5\\%, female—8\\%, and combined—14\\%. One-star consensus strategy has been used for final classification where performance improved significantly (more than 20\\% for all three types)}
    \end{conf-abstract}
        