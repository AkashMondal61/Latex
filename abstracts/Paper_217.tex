
    \begin{conf-abstract}[]
        {\textbf{Exploring Deep Learning Architectures for Eye Tracking Analysis in Autism Spectrum Disorder Detection}}
        {\textit{krishna sai koppula$^{1}$, Anupam Agrawal$^{2}$}}
        {$^{1}$IIIT Allahabad $\bullet$ $^{2}$IIIT Alahabad}
        {\texttt{mit2021046@iiita.ac.in, anupam@iiita.ac.in}}
        \indexauthors{koppula!krishna sai, Agrawal!Anupam}
        { Early detection of Autism Spectrum Disorder (ASD) is cru- cial for effective intervention and support. This study investigates the potential of utilizing eye tracking data and deep learning architectures for accurate ASD detection. Our research focuses on employing three deep learning models: a 4-layer Convolutional Neural Network (CNN), a pretrained ResNetv2, and a pretrained Visual Transformer, to ana- lyze visual scan path images derived from eye tracking data.The deep learning models are trained and evaluated using a rigorous 5-fold cross- validation approach. When trained on data augmented visual scan paths of eye tracking images, the 4-layer CNN achieves an accuracy of 0.95 and an Area Under the Curve (AUC) of 0.98, while the pretrained ResNetv2 model achieves an accuracy of 0.98 with an AUC of 0.81. Notably, the pretrained Visual Transformer model surpasses the others, attaining an accuracy of 0.989 and an outstanding AUC of 0.99, outperforming pre- vious research methods.The findings highlight the effectiveness of deep learning architectures in analyzing visual scan paths for ASD detection. The achieved accuracies and AUC values signify the potential of these models for reliable and precise identification of autism.}
    \end{conf-abstract}
        