
    \begin{conf-abstract}[]
        {\textbf{Testing of MEDA-based Biochip: A Proposed Technique for Functional Testing of Symmetric Set of Modules}}
        {\textit{Pranab Roy$^{1}$}}
        {$^{1}$J.K. Laxmipat University,Jaipur,Rajasthan}
        {\texttt{ronmarine14@yahoo.co.in}}
        \indexauthors{Roy!Pranab}
        {The MEDA biochip technology allows for the miniaturization of laboratory procedures and the automation of complex tasks of biomedical analysis, making it a valuable tool in many fields, including medicine, biology, and chemistry. One of the advantages of MEDA biochips is their ability to perform multiple analyses simultaneously, which can eventu-ally save time and resources. Due to its higher complexity and parallel execution of mul-tiple bioassays conventional testing methods used for conventional Biochips may not be adequate for fault diagnosis and detection in such devices. In this paper, we proposed new testing techniques for a given symmetric set of modules compatible with higher complexity and applicable for functional testing of MEDA-based biochips. We proposed a functional testing technique involving merging, mixing, and splitting of droplets (both horizontal and diagonal) for a set of modules of a given configuration located symmet-rically within a 2D grid MEDA biochip layout. We simulated the proposed testing tech-nique with the symmetric arrangement of modules with different numbers. Thereby we developed a specific test schedule for different module arrangements resulting in the utilization of optimal test resources, timing, and better module coverage.}
    \end{conf-abstract}
        