
    \begin{conf-abstract}[]
        {\textbf{CervixNet: A reward-based weighted ensemble framework for cervical cancer classification}}
        {\textit{Kaushiki Roy$^{1}$, Debapriya Banik$^{2}$, Ondrej Krejcar$^{3}$, Ram Sarkar$^{4}$, Deboparna  Bhattacharjee$^{5}$}}
        {$^{1}$Jadavpur University $\bullet$ $^{2}$Jadavpur University $\bullet$ $^{3}$University of Hradec Kralove $\bullet$ $^{4}$Jadavpur University $\bullet$ $^{5}$Jadavpur University}
        {\texttt{kaushiki.cse@gmail.com, debu.cse88@gmail.com, ondrej.krejcar@uhk.cz, ramjucse@gmail.com, deboparna11@gmail.com}}
        \indexauthors{Roy!Kaushiki, Banik!Debapriya, Krejcar!Ondrej, Sarkar!Ram, Bhattacharjee!Deboparna }
        {Cervical cancer is one of the most common causes of death among women worldwide. However, this fatal disease can be treated, and the mortality rate can be decreased if detected at an early stage. The Papanicolaou test is the gold standard for screening cervical cancer patients. However, the process of manual inspection is tedious and subject to manual errors. Thus, computer-based automatic screening is considered a viable alternative. To this end, in the present work, we have developed a novel system for classifying cervical cancer images as either normal or malignant. We have used the Herlev dataset in this study, which is the standard Pap smear image benchmark dataset. The proposed framework uses multiple base deep learning frameworks, namely Vision Transformer, Xception, VGG-19, VGG-16, and ResNet-101, for classification. Further, we have introduced a novel reward-based weighing technique to decide the weights of individual classifiers, which are, in turn, used to make decisions about the final class label of an input image using the weighted average technique. The proposed framework achieves an overall accuracy, precision, recall, F1-score, and AUC of 96\%, 94\%, 92\%, 93\%, and 94\%, respectively, for the binary classification task that is normal vs. malignant. An elaborate study of the performance achieved by our proposed framework on the Herlev dataset shows it to be both robust and effective for the Pap smear image classification task.}
    \end{conf-abstract}
        