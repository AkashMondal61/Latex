
    \begin{conf-abstract}[]
        {\textbf{Ensemble Learning of Maize Leaves Infected by Fall Armyworms using CNN}}
        {\textit{Soumadeep Bagui$^{1}$, Dipak kole$^{2}$, Anilabha  Datta$^{3}$, Avishek  Chatterjee$^{4}$, Kusal Roy$^{5}$, Sourav Bhunia$^{6}$}}
        {$^{1}$Jalpaiguri Government Engineering College $\bullet$ $^{2}$jgec $\bullet$ $^{3}$Jalpaiguri Government Engineering College $\bullet$ $^{4}$Jalpaiguri Government Engineering College $\bullet$ $^{5 2}$Bidhan Chandra Krishi Viswavidyalaya $\bullet$ $^{6}$JGEC}
        {\texttt{sb2310@cse.jgec.ac.in, dipak.kole@cse.jgec.ac.in, ad2307@cse.jgec.ac.in, ac2301@cse.jgec.ac.in, roy.kusal@bckv.edu.in, sb2592@cse.jgec.ac.in}}
        \indexauthors{Bagui!Soumadeep, kole!Dipak, Datta!Anilabha , Chatterjee!Avishek , Roy!Kusal, Bhunia!Sourav}
        {One of the most commercially successful cereal crops farmed worldwide is maize (Zea mays L.), which is a common staple food in many developing nations. Massive yield loss is caused by the extreme outbreak of fall armyworm in maize. The goal of this work is to create a deep learning-based model that is trained on images of healthy and fall armyworm infested maize leaves from a dataset using Convolutional Neural Network (CNN) architecture and enhance the accuracy using Ensemble learning. The Ensemble model is developed using keras functional API via training, Sequential model and two functional models that is ResNet-50 and VGG-16 to achieve a higher validation accuracy of 98.58 \%. Therefore, the detection of fall armyworm infested maize leaves and the treatment of fall armyworms may result in an increase in crop production.}
    \end{conf-abstract}
        