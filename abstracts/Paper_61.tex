
    \begin{conf-abstract}[]
        {\textbf{Unravelling the Network Landscape: A Comparative Analytical Approach to Investigate Protein-Protein Interaction Networks in Normal v/s Tumour Cells}}
        {\textit{Manoswita Bose$^{1}$, Neepa Biswas$^{2}$, Dhrubasish Sarkar$^{3}$}}
        {$^{1}$Amity University, Kolkata $\bullet$ $^{2}$Jadavpur University $\bullet$ $^{3}$Supreme Institute of Management and Technology"}
        {\texttt{simabose12@gmail.com, biswas.neepa@gmail.com, dhrubasish@inbox.com}}
        \indexauthors{Bose!Manoswita, Biswas!Neepa, Sarkar!Dhrubasish}
        {Protein-Protein Interaction (PPI) networks are complex networks that model the interactions between proteins. Various biological processes, such as signal transduction, gene regulation, and metabolism, heavily depend on these interactions. In addition, they are also important targets for drug discovery. Network analysis, a computational approach that characterises the topological properties of PPI networks, has become a powerful tool for comprehending the organisation and function of these networks. These techniques can be applied to PPI networks to gain insights into the functional modules, key proteins, and biological pathways that are involved in these processes.  In this paper, two sets of PPI networks - one representing normal cells and the other representing tumour cells - were firstly analysed using the NetworkX library in Python, followed by their visualisation using Matplotlib and Seaborn libraries of Python and Gephi. The networks were constructed from publicly available protein interaction data and network analysis techniques were subsequently used to compare their properties. Additionally, the results may provide insights into the underlying biological processes. To sum up, the analysis in this paper demonstrates the utility of network analysis in understanding the differences between normal and tumour cells at the protein interaction level. }
    \end{conf-abstract}
        