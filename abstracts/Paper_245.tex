
    \begin{conf-abstract}[]
        {\textbf{An Efficient Clustering Algorithm on Next-Generation Sequence Data}}
        {\textit{Manan Kr Gupta$^{1}$, Soumen  Kumar Pati$^{2}$}}
        {$^{1}$Maulana Abul Kalam Azad University of Technology $\bullet$ $^{2}$Maulana Abul Kalam Azad University of Technology, West Bengal}
        {\texttt{mownon89@gmail.com, soumenkrpati@gmail.com}}
        \indexauthors{Gupta!Manan Kr, Pati!Soumen  Kumar}
        {The clustering algorithms are an unsupervised machine learning methodology widely utilized in various fields to find and identify patterns in data. In bioinformatics, clustering plays a crucial role that primarily helps to find relations between similar gene sequences. In this work, a new clustering technique has been proposed that uses the concept of tensors, Riemannian manifolds, and geodesics to cluster a given set of biological data. This new method of clustering Next Generation Sequence data has been developed keeping in mind that it should yield previously undetected patterns in such type of data. The proposed work is compared with well-known clustering techniques such as k-means, DBSCAN, hierarchical and fuzzy c-means clustering algorithms via several metrics (DB index, Dunn index, Rand score, and Silhouette score) and shows its efficiency.  }
    \end{conf-abstract}
        