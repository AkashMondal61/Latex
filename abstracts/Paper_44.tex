
    \begin{conf-abstract}[]
        {\textbf{A 2D Based Synthesis Strategy for Nearest Neighbor  Transformation of Quantum Circuits}}
        {\textit{Anirban Bhattacharjee$^{1}$}}
        {$^{1}$IIEST Shibpur}
        {\texttt{anirbanbhattacharjee330@gmail.com}}
        \indexauthors{Bhattacharjee!Anirban}
        {In past several years, quantum computing has made enormous strides, and it is already starting to create scientific footprints in the manufacturing sector. The Nearest Neighbour (NN) condition, which requires the operational qubits associated with the quantum gates to be next to one another, is one such design limitation that has been encountered despite the significant breakthroughs in the practical realization of quantum circuits. In this paper, we have illustrated a heuristic based design methodology to effectively convert quantum circuits into nearest neighbor-compliant architectures. Our proposed approach involves fitting the algorithm into a 2D design. We extensively tested our algorithm across a diverse range of benchmarks, and comparisons with leading design approaches revealed significant enhancements. Our overall technique is built on our initial qubit placing strategy, due to which we proposed an effective traversing and mapping system that is carried out with the help of modified prim's algorithm and genetic algorithm, respectively. This combination of heuristic strategies has shown promising outcomes in transforming quantum circuits into NN compliant architectures. In conclusion, we thoroughly evaluated a wide array of benchmarks using our conversion techniques and conducted a comprehensive comparative analysis, including state-of-the-art designs and various placement methods. We compared the efficacy of our proposed approach with previous research efforts.}
    \end{conf-abstract}
        